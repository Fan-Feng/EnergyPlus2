\chapter{CSVproc}\label{csvproc}

This simple post processing utility may be useful when doing parametric analyses. It takes a CSV (comma separated values file) and performs some simple statistics. It is a very small application with no interface. It is typically executed from the command line.

\begin{enumerate}
\def\labelenumi{\arabic{enumi})}
\item
  Open a DOS command prompt window (Start > Programs > Accessories > Command Prompt)
\item
  Change to the directory where EnergyPlus is installed (modify the commands below if you did not install EnergyPlus in the default install path):
\end{enumerate}

C:

CD \textbackslash{}\textless{}root folder\textgreater{}\textless{}/span\textgreater{}

\begin{enumerate}
\def\labelenumi{\arabic{enumi})}
\setcounter{enumi}{2}
\tightlist
\item
  Change to the specific folder for the coefficient conversion applications:
\end{enumerate}

CD PostProcess

\begin{enumerate}
\def\labelenumi{\arabic{enumi})}
\setcounter{enumi}{3}
\tightlist
\item
  Run the program:
\end{enumerate}

CSVproc \textless{}filename\textgreater{}

Where \textless{}filename\textgreater{} is the name of a CSV file, including extension. There is a simple readme.txt file in the folder. The program performs some simple statistics on each column and creates a new file with the same name without extension and -PROC.CSV added to the name.

The statistics performed on each column are:

\begin{itemize}
\item
  SUM
\item
  MAX
\item
  MIN
\item
  AVERAGE
\item
  COUNT
\item
  COUNTIF \textgreater{} 0
\item
  COUNTIF \textgreater{} 5
\item
  COUNTIF \textgreater{} 10
\item
  COUNTIF \textgreater{} 15
\item
  COUNTIF \textgreater{} 20
\item
  COUNTIF \textgreater{} 25
\item
  COUNTIF \textgreater{} 30
\item
  COUNTIF \textgreater{} 35
\item
  COUNTIF \textgreater{} 40
\item
  COUNTIF \textgreater{} 45
\item
  COUNTIF \textgreater{} 50
\item
  COUNTIF \textgreater{} 55
\item
  COUNTIF \textgreater{} 60
\item
  COUNTIF \textgreater{} 65
\item
  COUNTIF \textgreater{} 70
\item
  COUNTIF \textgreater{} 75
\item
  COUNTIF \textgreater{} 80
\item
  COUNTIF \textgreater{} 85
\item
  COUNTIF \textgreater{} 90
\item
  COUNTIF \textgreater{} 95
\item
  COUNTIF \textgreater{} 100
\item
  COUNTIF = 1
\item
  COUNTIF \textless{} 19.9
\item
  COUNTIF \textgreater{} 24.0
\end{itemize}

Obviously, not all statistics are relevant for every output report variable. The average is based on the sum divided by the number of non-blank rows. The average is not based on the length of time for that timestep. Due to this, CSVproc is best suited for an hourly output file.

Source code is available upon request from jglazer@gard.com.
